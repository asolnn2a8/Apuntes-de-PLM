% macros
\newcommand{\calI}{\mathcal I}
\newcommand{\Proy}{\mathrm Proy}
\newcommand{\calQ}{\mathcal Q}
\newcommand{\calT}{\mathcal T}
\newcommand{\calL}{\mathcal L}
\newcommand{\ALG}{\mathrm{ALG}}
\newcommand{\alg}{\mathrm{alg}}
\newcommand{\opt}{\mathrm{opt}}
\newcommand{\conv}{\mathrm{conv}}
\newcommand{\BB}{\mathbb B}
\newcommand{\QQ}{\mathbb Q}
\newcommand{\RR}{\mathbb R}
\newcommand{\NN}{\mathbb N}
\newcommand{\ZZ}{\mathbb Z}
\newcommand{\CC}{\mathbb C}
\newcommand{\OPT}{\mathrm{OPT}}
\newcommand{\Ord}[1][n]{\ensuremath{ \mathcal{O}(#1) }}
\DeclareMathOperator{\conv}{conv}

%Teoremas, Lemas, etc.
\theoremstyle{plain}
\newtheorem{teo}{Teorema}
\newtheorem{lem}{Lema}
\newtheorem{prop}{Proposición}
\newtheorem{cor}{Corolario}
\newtheorem{obs}{Observaci\'on}

\theoremstyle{definition}
\newtheorem{defi}{Definición}
\newtheorem{eje}{Ejemplo}
\newtheorem{ejer}{Ejercicio}

\newcommand{\Julia}{\textsc{ampl}\xspace}
\newcommand{\CPLEX}{\textsc{cplex}\xspace}
\newcommand{\GUROBI}{\textsc{gurobi}\xspace}
\newcommand{\JULIA}{\textsc{julia}\xspace}
\newcommand{\MINOS}{\textsc{minos}\xspace}
\newcommand{\CBC}{\textsc{cbc}\xspace}

\newcommand{\exe}[1]{{\bfseries\slshape\ttfamily#1}}
\newcommand{\Hilight}{\color{cyan}\rule[-4pt]{0.65\linewidth}{14pt}}

% Comando para notas al margen
\def\NAM#1#2{
\ifnum \notasalmargen=1 #1 \else #2 \fi
}

% =========== Clase 3 ===============
\def\NoNumber#1{{\def\alglinenumber##1{}\State #1}\addtocounterhh{ALG@line}{-1}}
\newlength{\algofontsize}
\setlength{\algofontsize}{6pt}

\definecolor{light_gray}{rgb}{0.7,0.7,0.7}
\makeatletter
\newcommand\currentStyle@lstparam{}
\lst@AddToHook{Output}{\global\let\currentStyle@lstparam\lst@thestyle}
\lst@AddToHook{OutputOther}{\global\let\currentStyle@lstparam\lst@thestyle}
\makeatother

\makeatletter
% Usage: \highlightcode{color}{content}
\newcommand{\highlightcode}[2]{\currentStyle@lstparam \textcolor{#1}{#2}}
\newcommand{\bl}[1]{\highlightcode{blue}{#1}}
\makeatother
% =========== Clase 3 ===============
% fin macros